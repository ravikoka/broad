%\documentclass[../main.tex]{subfiles}

\begin{utabstract}
    In high energy heavy-ion collisions at the Large Hadron Collider, the temperatures and densities necessary to form the quark-gluon plasma (QGP) are reached. This QGP is an state of matter in which quarks and gluons---ordinarily bound inside color-neutral hadrons---are deconfined. The QGP is believed to have filled the universe microseconds after the Big Bang, when the universe was extremely hot. Strangeness enhancement, an important signature of QGP formation, is observed in p-Pb collisions, calling into question the limits of QGP formation in small systems. To untangle strangeness production in p-Pb, we can utilize two-particle hadron---$\Lambda$ angular correlations to separate jet-like yields from strangeness production in a possible QGP medium. The yields and widths of these correlations are compared to h---h as a baseline. We evaluate whether these hadron---$\Lambda$ and h---h correlations are affected by the finite acceptance of ALICE, using $10^7$ proton-proton PYTHIA6 monte carlo events at $\sqrt{s}=7$ TeV. We calculate the yields and widths corresponding to the near- and away-side peak of our hadron---$\Lambda$ and h---h correlations across various kinematic cuts, and calculate the corresponding ratios (hadron---$\Lambda$ over h---h). We find agreement between the yields and yield ratios across all acceptances. In addition, we find a broad agreement in the widths of the correlations, and no deviations in the width ratios. This suggests our h--$\Lambda$ and h---h correlations require no extra systematic corrections due to finite acceptance. The code used for this analysis is available at \url{https://github.com/ravikoka/broad}
\end{utabstract}