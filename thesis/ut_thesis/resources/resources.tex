\documentclass[../main.tex]{subfiles}

\begin{document}
This is a list of useful resources for undergraduates beginning their journey in heavy-ion physics. 

\begin{itemize}
    \item \textbf{CERN ROOT Software.} This is the software you will utilize for the actual analysis. There are many tutorials available. Learn to read the documentation, and it will help you later! URL: \url{https://root.cern/}
    \item \textbf{Nevis Labs ROOT Tutorial.} Great, comprehensive tutorial to learn ROOT. Covers both some aspects of ROOT in C++ and Python. \url{https://www.nevis.columbia.edu/~seligman/root-class/}
    \item \textbf{Physics Matters ROOT Tutorials.} A great playlist to learn ROOT from install to histogram arithmetic. URL: \url{https://www.youtube.com/watch?v=KPz-dNjdx40&list=PLLybgCU6QCGWLdDO4ZDaB0kLrO3maeYAe}
    \item \textbf{CERN Courier.} Subscribe to the magazine. URL: \url{https://cerncourier.com/}
    \item \textbf{InspireHEP.} A place for high energy physicists to share articles, list conferences, and more. This is really just a good place to check what people are publishing and find articles. URL: \url{https://inspirehep.net/}
    \item \textbf{arXiv.} A magical place in physics where pre-prints are uploaded. This is the spirit of open-access. URL: \url{https://arxiv.org/} 
    \item \textbf{Ultrarelativistic Heavy-Ion Collisions by Ramona Vogt.} A great introductory book to heavy-ion collisions. The material is hard, but certain sections are enriching. 
    \item \textbf{Phenomenology of Ultra-Relativistic Heavy-Ion Collisions by Wojciech Florkowski.} A more advanced book on heavy-ion collisions. The first chapter is a great introduction with links to experiment. 
    \item \textbf{Modern Particle Physics by Mark Thomson.} A superb introduction to modern particle physics with basic explanations of colliders, Feynman diagrams, and theory. If you've taken Quantum II, this is a great resource. 
    \item \textbf{Detectors in Particle Physics: A Modern Introduction by Viehhauser and Weidberg.} If you're tired of number crunching, read this book. It covers experimental methods in particle physics in great detail and with broad scope. URL: \url{https://inspirehep.net/literature/2767057}
    \item \textbf{Particle Data Group.} For basic figures, PDG codes, and measured parameters. URL: \url{https://pdg.lbl.gov/}
    \item \textbf{Spacetime Physics by Taylor and Wheeler.} A good introduction to special relativity. I haven't read this book, but I've heard good things. URL: \url{https://www.eftaylor.com/spacetimephysics/}
    \item \textbf{A Beginners Course in General Relativity by Bernard Schutz.} The first chapter has a good, brief coverage of special relativity if you are already somewhat familiar with the subject.
    \item \textbf{Quantum Mechanics: A Paradigms Approach by David McIntyre.} This is a superb book for quantum mechanics. Most of my intuition for quantum phenomena from the idea of an abstract state to perturbation theory comes from this book. The exercises are great too. I cannot recommend it enough. 
    \item \textbf{Heavy-Ion Collisions: The Big Picture, and the Big Questions.} For a big-picture overview of the field. This review is dense for someone just starting, but it would be beneficial to take a peek at some point. URL: \url{https://arxiv.org/abs/1802.04801}.
\end{itemize}


\end{document}