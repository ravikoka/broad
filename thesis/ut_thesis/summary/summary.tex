\documentclass[../main.tex]{subfiles}

\begin{document}
In high energy heavy-ion collisions, the temperatures and densities necessary to create the quark-gluon plasma (QGP) are reached. This is a state of matter where the partons ordinarily bound inside color-neutral hadrons are deconfined. The QGP is thought to have filled the universe microseconds after the Big Bang and provides a probe of quantum chromodynamics (QCD) at extreme energy densities. While physicists expected QGP formation in relativistic heavy-ion collisions, we did not expect a medium in small systems. However, measured collective effects in these small systems indicate possible droplets of QGP. To study the limits of QGP formation, we focus on strangeness enhancement, the relative increase in production of strange quarks (compared to up and down quarks) as system size increases. This is an important signature of QGP production. High multiplicity p-Pb events reach the level of strangeness enhancement seen in low-multiplicity Pb-Pb, demanding an investigation of strangeness production in p-Pb. An analysis on behalf of the ALICE collaboration used h---$\Lambda$ angular correlations to separate production of strange quarks in and out of jets. Due to the finite acceptance of ALICE, this required a pseudorapidity cut of $|\eta|<0.8$. In this thesis, we evaluate the effect of acceptance on the widths and yields of h---$\Lambda$ and h---h correlations. Using p-p PYTHIA6 events at $\sqrt{s}=7$ TeV, we find no significant effects from the limited acceptance of ALICE. The yields agree for all $\eta$ cuts. Similarly, the widths are evaluated using multiple fit functions, which show broad agreement. The ratios of these widths and yields, calculated as h---$\Lambda$ over h---h (strange over non-strange), are consistent across all fits and acceptances. 
\end{document}
